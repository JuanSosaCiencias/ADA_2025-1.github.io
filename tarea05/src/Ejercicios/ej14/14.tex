\textbf{Encuentre la parentización óptima para multiplicar seis matrices de dimensiones $4 \times 9$, $9 \times 4$, $4 \times 10$, $10 \times 2$, $2 \times 5$, $5 \times 6$.}\vspace{.2cm}

Primero que nada voy a ponerle nombres a las matrices con las que estoy trabajando para que sea más fácil referirse a ellas. Así que las matrices serán:

\begin{align*}
    A0 & : 4 \times 9 \\
    A1 & : 9 \times 4 \\
    A2 & : 4 \times 10 \\
    A3 & : 10 \times 2 \\
    A4 & : 2 \times 5 \\
    A5 & : 5 \times 6   
\end{align*}

Y el vector de dimensiones será $p = \{4, 9, 4, 10, 2, 5, 6\}$. \vspace{.2cm}

\textcolor{bibi}{Usando DP}
\begin{quote}
    Voy a crear una matriz $dp$ de tamaño $6 \times 6$ donde $dp[i][j]$ va a ser el costo mínimo de multiplicar las matrices $A_i \times A_{i+1} \times \ldots \times A_j$. \vspace{.2cm}

    Además, voy a crear una matriz $s$ de tamaño $6 \times 6$ donde $s[i][j]$ va a ser el índice de la matriz que se va a multiplicar en la última multiplicación de la cadena $A_i \times A_{i+1} \times \ldots \times A_j$. \vspace{.2cm}

    Para llenar la matriz $dp$ y $s$ voy a usar la siguiente función:
    \begin{align*}
        \begin{cases}
            \text{dp}[i][j] &= \min_{i \leq k < j} \{ \text{dp}[i][k] + \text{dp}[k+1][j] + p_{i} \cdot p_{k+1}\cdot p_{j+1} \} \quad \text{si } i < j \\
            \text{dp}[i][j] &= 0 \quad \text{si } i = j
        \end{cases} \\
    \end{align*}

    Donde $p_{i-1} \cdot p_k \cdot p_j$ es el numero de operaciones que se hacen al multiplicar las matrices resultantes de esas subcadenas. \vspace{.2cm}

    Y para llenar la matriz $s$ voy a usar la siguiente función:
    \begin{align*}
        \text{s}[i][j] = k
    \end{align*}

    Empezamos por llenar la tabla $dp$ por su diagonal, es decir, $dp[i][i] = 0$ para todo $i$ esto es porque el costo mínimo de multiplicar una sola matriz es 0, despues seguimos la función es optimo ir llendo en diagonales empezando por la principal para arriba, podemos optimizar y solo llenar la mitad de la matriz ya que es simétrica. \vspace{.2cm}

    Ahora si vamos a aplicarlo y llenar la tabla (cuidado con confundir indices y elementos): \vspace{.2cm}

    \begin{center}
        % Tabla de costos (dp)
        \noindent
        \textbf{Tabla de costos mínimos (dp)} \\
        \begin{tabular}{c|*{6}{c}}
        i/j & $0$ & $1$ & $2$ & $3$ & $4$ & $5$ \\
        \hline
        $0$ &  0  & 144 & 304 & 224 & 264 & 332 \\
        $1$ &     &  0  & 360 & 152 & 242 & 320 \\
        $2$ &     &     &  0  & 80  & 120 & 188 \\
        $3$ &     &     &     &  0  & 100 & 180 \\
        $4$ &     &     &     &     &  0  & 60  \\
        $5$ &     &     &     &     &     & 0   \\
        \end{tabular}

        \vspace{1cm}

        % Tabla de divisiones (s)
        \noindent
        \textbf{Tabla de puntos de división (s)} \\
        \begin{tabular}{c|*{6}{c}}
        i/j & $0$ & $1$ & $2$ & $3$ & $4$ & $5$ \\
        \hline
        $0$ &  0  &  0  &  1  &  0  &  3  &  3  \\
        $1$ &     &  0  &  1  &  1  &  3  &  3  \\
        $2$ &     &     &  0  &  2  &  3  &  3  \\
        $3$ &     &     &     & 0   &  3  &  3  \\
        $4$ &     &     &     &     &  0  &  4  \\
        $5$ &     &     &     &     &     &  0  \\
        \end{tabular}
    \end{center}

    No escribi todos los calculos para no hacerlo tan largo pero es un proceso tedioso y cada uno de los valores puede cambiar dependiendo del k en el que vamos. \vspace{.2cm}

    Como vemos el costo minimo es 332 y el punto de división es 3, eso significa que creamos un grupo con las matrices $A_0, A_1, A_2,A_3$ y otro con las matrices $A_4, A_5$ nos queda $(A_0, A_1,$ $ A_2,A_3)$ $(A_4$, $ A_5)$ hasta ahora. \vspace{.2cm} 

    Checamos $s[0,3]=0$ asi que el primer grupo se divide en $(A_0$ $(A_1, A_2,A_3))$ y el segundo grupo se queda igual. \vspace{.2cm}  

    Checamos $s[1,3]=1$ asi que el primer grupo se divide en $(A_0$ $(A_1$ $(A_2,A_3)))$ y el segundo grupo se queda igual, separar mas este grupo ya no tiene mucho efecto porque se queda igual. \vspace{.2cm}

    Seguimos con el segundo grupo pero igual ya solo son 2 matrices entonces lo dejamos asi, al final nos queda la siguiente parentización óptima: $$(A_0 \times (A_1 \times (A_2 \times A_3))) \times (A_4 \times A_5)$$ \vspace{.2cm}
\end{quote}

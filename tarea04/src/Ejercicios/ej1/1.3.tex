\textbf{Monedas Portuguesas 1, 2.5, 5, 10, 20, 25, 50 escudos}\vspace{.2cm}

\textcolor{bibi}{Contraejemplo}
\begin{quote}
    En este caso también es obvio que no va a funcionar el algoritmo glotón, ya que si por ejemplo se quiere dar el cambio de 40 unidades, primero se dara una moneda de 25, una de 10 y una de 5, en total 3 monedas. Sin embargo, si se diera 2 monedas de 20, se tendría un total de 2 monedas, lo cual es menor que 3. \vspace{.2cm}

    Si se intenta seguir la idea de la demostración 1a, llegariamos a una contradiccion durante la induccion, asi demostrando que el algoritmo glotón no es optimo. \vspace{.2cm}
\end{quote}
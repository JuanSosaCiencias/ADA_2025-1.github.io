\textbf{Monedas Inglesas 30, 24, 12, 6, 3, 1, 1/2 y 1/4 peniques}\vspace{.2cm}

\textcolor{bibi}{Contraejemplo}
\begin{quote}
    En este caso es claro que existe un contraejemplo en donde el algoritmo no funciona, por ejemplo, si se tiene que regresar 48 peniques, el algoritmo daría 30, 12, 6, lo cual no es la mejor opción, ya que se pueden dar 24, 24. \vspace{.2cm}

    Si se intentara demostrar con la idea del 1a, llegarias a la contradiccion durante la induccion, ya que las soluciones optimas de los subproblemas no siempre lleva a una solucion optima. \vspace{.2cm}
\end{quote}
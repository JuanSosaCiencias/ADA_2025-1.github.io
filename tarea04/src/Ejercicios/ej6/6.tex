\textbf{
    El profesor López tiene 2 hijos los cuales no se llevan nada bien. Los chiquillos se 
    odian tanto que no sólo se niegan a caminar juntos a la escuela, si no que además se 
    niegan a caminar en cualquier acera en la que el otro hermano haya puesto pie ese día.
    Los chiquillos no tienen problemas con que sus caminos coincidan en algunas esquinas.
    Afortunadamente, tanto la casa del profesor como la escuela están en una esquina, fuera
    de eso el profesor no está seguro si será posible meter a los 2 hijos en la misma escuela.
    Muestre cómo modelar el problema de decidir si es posible enviar a los 2 hijos a la misma 
    escuela como un problema de flujos.
}\vspace{.2cm}

\textcolor{bibi}{Modelando el problema como un problema de flujos:}
\begin{quote}
    Comenzamos por modelar las intersecciones de las calles como nodos, debido a que pueden pasar ambos hijos no tienen restricciones, ademas, el nodo de la casa del profesor es el nodo fuente y el nodo de la escuela es el nodo sumidero. \vspace{.2cm}

    Ahora agregaremos las aceras como aristas, diremos que tienen capacidad de 1 pues si ya paso uno de los 2 el otro no puede usarla, unimos las intersecciones con las aceras que los conectan. \vspace{.2cm}

    Ahora tenemos un grafo dirigido con capacidad en las aristas, los nodos fuentes y sumideros deben tener al menos 2 aristas salientes y entrantes respectivamente para que el flujo pueda pasar por ellos. \vspace{.2cm}
    
    Para decidir si es posible que ambos hijos lleguen a la escuela sin violar restricciones, basta con encontrar el flujo máximo en el grafo, si el flujo máximo es 2 entonces es posible que ambos hijos lleguen a la escuela, de lo contrario no es posible, esto porque existen 2 caminos disjuntos entre la casa del profesor y la escuela que puede tomar cada hijo.\vspace{.2cm}

    Para encontrar el flujo máximo podemos usar el algoritmo de Ford-Fulkerson o cualquier otro algoritmo de flujo máximo. \vspace{.2cm}
\end{quote}
\textbf{Dise\~na un algoritmo de programaci\'on din\'amica eficiente que determine si $Z$ es un \textit{shuffle} de $X$ y $Y$. \textit{Hint}: Los valores de la matriz de programaci\'on din\'amica que construyas, podr\'ian ser valores booleanos y no num\'ericos.}\vspace{.2cm}

El problema se parece bastante a la subcadena creciente mas grande.
La idea es usar una matriz de $|X|+1$ filas y $|Y|+1$ columnas, 
donde la celda $(i,j)$ indica si los primeros $i+j$ caracteres de $Z$
son un \textit{shuffle} de los primeros $j$ caracteres de $X$ y 
los primeros $i$ caracteres de $Y$. (podemos cambiar cual es quien pero asi lo hice) \vspace{.2cm}
    
\textcolor{bibi}{Usando matriz y programaci\'on din\'amica:}\vspace{.2cm}
\begin{quote}

    Podemos empezar verificando que $|X|+|Y|=|Z|$, si no es asi entonces
    no puede ser un \textit{shuffle} de $X$ y $Y$. \vspace{.2cm}

    Tenemos entonces que nuestra matriz se va a ver algo as\'i:

    \begin{table}[H]
        \centering
        \begin{tabular}{lllllll}
                &                              &                       & $x_0$                 & $x_1$                 & $\dots$               & $x_n$                 \\
                & i/j                          & 0                     & 1                     & 1                     & $\dots$               & n                     \\ \cline{3-7} 
                & \multicolumn{1}{l|}{0}       & \multicolumn{1}{l|}{1} & \multicolumn{1}{l|}{} & \multicolumn{1}{l|}{} & \multicolumn{1}{l|}{} & \multicolumn{1}{l|}{} \\ \cline{3-7} 
        $y_0$   & \multicolumn{1}{l|}{1}       & \multicolumn{1}{l|}{} & \multicolumn{1}{l|}{} & \multicolumn{1}{l|}{} & \multicolumn{1}{l|}{} & \multicolumn{1}{l|}{} \\ \cline{3-7} 
        $y_1$   & \multicolumn{1}{l|}{2}       & \multicolumn{1}{l|}{} & \multicolumn{1}{l|}{} & \multicolumn{1}{l|}{} & \multicolumn{1}{l|}{} & \multicolumn{1}{l|}{} \\ \cline{3-7} 
        $\dots$ & \multicolumn{1}{l|}{$\dots$} & \multicolumn{1}{l|}{} & \multicolumn{1}{l|}{} & \multicolumn{1}{l|}{} & \multicolumn{1}{l|}{} & \multicolumn{1}{l|}{} \\ \cline{3-7} 
        $y_m$   & \multicolumn{1}{l|}{m}       & \multicolumn{1}{l|}{} & \multicolumn{1}{l|}{} & \multicolumn{1}{l|}{} & \multicolumn{1}{l|}{} & \multicolumn{1}{l|}{} \\ \cline{3-7} 
        \end{tabular}
    \end{table}

    Ahora vamos a ver como llenarla, primero por la definición que hicimos, el indice
    (i,j) sera 1 pues los primeros 0+0 caracteres de Z siempre seran shuffle de los primeros
    0 caracteres de X y los primeros 0 caracteres de Y. \vspace{.2cm}

    Para la primera fila, hay que checar que los primeros $j$ caracteres de $X$ sean iguales
    a los primeros $j$ caracteres de $Z$, si es asi, entonces la celda $(0,j)$ sera 1, si no
    entonces sera 0; para checar esto podemos checar si el caracter j es igual en ambos y despues
    checar si a la izquierda ya tenemos un 1. ( esto es equivalente a: M[0][j]=M[0][j-1] AND X[j-1]==Z[j-1]
    tenemos que restar 1 porque las cadenas tienen indice en 0 pero tambien lo puedes entender con cantidad
    de caracteres) \vspace{.2cm}

    Para la primera columna, es lo mismo que la fila pero con $Y$ y $Z$, esto se puede entender como
    M[i][0]=M[i-1][0] AND Y[i-1]==Z[i-1]. \vspace{.2cm}

    Ahora para el caso de en medio hay que checar que los primeros $i+j$ caracteres de $Z$ sean shuffle
    de los primeros $j$ caracteres de $X$ y los primeros $i$ caracteres de $Y$, esto se puede entender
    como M[i][j]=M[i-1][j] AND Y[i-1]==Z[i+j-1] OR M[i][j-1] AND X[j-1]==Z[i+j-1]. \vspace{.2cm}

    De manera general nuestra funcion quedaria algo asi:
    \scriptsize
    \begin{align*}
        M[i][j]=\begin{cases}
            1 & \text{si } i=0 \text{ y } j=0 \\
            (M[i-1][j] \text{ AND } Y[i-1]==Z[i+j-1]) \text{ OR } (M[i][j-1] \text{ AND } X[j-1]==Z[i+j-1]) & \text{si } i>0 \text{ y } j>0 \\
        \end{cases}
    \end{align*}

    Al final sabremos si $Z$ es un \textit{shuffle} de $X$ y $Y$ si $M[|Y|][|X|]=1$, ademas el camino
    para ir de la celda $(|Y|,|X|)$ a la celda $(0,0)$ nos dira cuales son los caracteres que se usaron. \vspace{.2cm}

    Este algoritmo tiene complejidad $O(|X|*|Y|)$ y usa $O(|X|*|Y|)$ memoria, esto pues la matriz es de tamaño
    $(|X|+1)*(|Y|+1)$ y se llena en cada celda una vez tomando $O(1)$ tiempo (checar arriba y abajo, tomar un indice de 
    una cadena y compararlo con el indice en otra cadena que puede ser O(1)). \vspace{.2cm}

    Para entender mejor checar el ejemplo de arriba.
\end{quote}


\newpage
\textbf{Una empresa est\'a planeando una fiesta para sus empleados. Los organizadores de la fiesta quieren que sea una fiesta divertida, por lo que han asignado una calificaci\'on de ``diversi\'on'' a cada empleado. Los empleados est\'an organizados en una estricta jerarqu\'ia, es decir, un \'arbol enraizado en el presidente. Sin embargo, hay una restricci\'on en la lista de invitados a la fiesta: tanto un empleado como su supervisor inmediato (padre en el \'arbol) no pueden asistir a la fiesta (porque eso no ser\'ia divertido). Dise\~ne un algoritmo de tiempo lineal que haga una lista de invitados para la fiesta y que maximice la suma de las calificaciones de ``diversi\'on'' de los invitados.}\vspace{.2cm}

Lo primero que hay que notar es que no hay restricciones sobre la cantidad de hijos que puede tener un nodo en el \'arbol ni tampoco si el valor de ``diversi\'on'' es positivo o negativo; asi que vamos a ahcer unas aclaraciones previas. \vspace{.2cm}

\textbf{Aclaraciones:}
\begin{itemize}
    \item Solo vamos a considerar valores de ``diversi\'on'' positivos. (el ayudante Adri\'an autoriza aunque no ipmorta tanto para la soluci\'on)
    \item El arbol puede ser un \'arbol binario o no.
    \item Solo vamos a considerar arboles con mas de 2 nodos pues si no el problema es trivial. (igualmente se resuelve con el algoritmo propuesto)
\end{itemize}

La idea central del algoritmo va a ser utilizar el árbol que ya existe, agregarle información y recorrerlo de manera eficiente. \vspace{.2cm}

\textcolor{bibi}{Usando el árbol que ya existe + recorridos:} \vspace{.2cm}
\begin{quote}
    En este algoritmo, nos interesa empezar a procesar por las hojas del árbol, ya que si un nodo es hoja, no tiene hijos y por lo tanto no tiene restricciones. \vspace{.2cm}

    Podemos utilizar un recorrido en postorden para recorrer el árbol, de manera que procesamos primero los hijos de un nodo antes de procesar al nodo en sí, este recorrido tiene complejidad O(n). (la implementación especifica no nos interesa mas porque ni es un arbol binario xd) \vspace{.2cm}

    Ahora, en cada nodo vamos a agregarle información, para ello vamos a usar 2 funciones que se pueden describir asi: \vspace{.2cm}
    \begin{itemize}
        \item \textbf{Incluir:} Si incluimos al nodo en la fiesta, entonces sus hijos no pueden ir. Esto se ve asi:
        \begin{align*}
            \text{Incluir}(nodo) = nodo.diversion +\sum_{h \in hijos} \text{excluir}(h)
        \end{align*}
        \item \textbf{Excluir:} Si excluimos al nodo en la fiesta, entonces podemos o no incluir a sus hijos. Esto se ve asi: 
        \begin{align*}
            \text{Excluir}(nodo) = \sum_{h \in hijos} \max\{\text{incluir}(h), \text{excluir}(h)\}
        \end{align*}
    \end{itemize}

    Ademas de estas funciones, vamos a agregar la base de la recurrencia, como ya dijimos cuando un nodo es hoja no tiene restricciones, por lo que podemos definir las funciones asi: \vspace{.2cm}
    \begin{align*}
        \text{incluir}(nodo) &= nodo.diversion \\
        \text{excluir}(nodo) &= 0
    \end{align*}

    Adicionalmente, si queremos considerar negativos (aunque ya lo resuelve) podemos decir que si un valor es negativo simplemente no lo agregamos. (como en la vida real) \vspace{.2cm}

    Entonces para cada nodo, empezando por las hojas agregamos esta información y recuperamos la información de sus hijos para poder calcular la información del nodo padre. \vspace{.2cm}

    \textbf{Importante:} las funciones no llaman a la otra función y asi recursivamente si no que llaman al valor que tenemos guardado en el nodo. \vspace{.2cm}

    Entonces el arbol se va a ir llenando algo asi en el recorrido: \vspace{.2cm}

    \begin{center}
        \begin{figure}[H]
            \centering
            \resizebox{.55\textwidth}{!}{%  
            \begin{circuitikz}
                \tikzstyle{every node}=[font=\large]
                \draw  (5,6.5) circle (0cm);
                \draw  (5,6.5) circle (0.5cm) node {\LARGE 3} ;
                \draw  (7.5,6.5) circle (0.5cm) node {\LARGE 2} ;
                \draw  (10,6.5) circle (0.5cm) node {\LARGE 1} ;
                \draw  (12.5,6.5) circle (0.5cm) node {\LARGE \textbf{5}} ;
                \draw  (7.5,9) circle (0.5cm) node {\LARGE \textbf{6}} ;
                \draw  (12.5,9) circle (0.5cm) node {\LARGE \textbf{5}} ;
                \node [font=\LARGE] at (8,8.5) {};
                \draw  (7.5,11.5) circle (0.5cm) node {\LARGE \textbf{6}} ;
                \draw  (15,9) circle (0.5cm) node {\LARGE \textbf{12}} ;
                \draw  (17.5,9) circle (0.5cm) node {\LARGE \textbf{0}} ;
                \draw  (16.25,11.5) circle (0.5cm) node {\LARGE \textbf{8}} ;
                \draw  (12.5,11.25) circle (0.5cm) node {\LARGE \textbf{7}} ;
                \draw  (12.5,14) circle (0.5cm) node {\LARGE \textbf{12}} ;
                \draw  (13.75,16.5) circle (0.5cm) node {\LARGE \textbf{30}} ;
                \draw  (10,14) circle (0.5cm) node {\LARGE \textbf{0}} ;
                \draw  (15,14) circle (0.5cm) node {\LARGE \textbf{12}} ;
                \draw  (17.75,14) circle (0.5cm) node {\LARGE \textbf{2}} ;
                \draw [short] (5,6.5) -- (7.5,9);
                \draw [short] (7.5,6.5) -- (7.5,9);
                \draw [short] (10,6.5) -- (7.5,9);
                \draw [short] (12.5,6.5) -- (12.5,9);
                \draw [short] (12.5,9) -- (12.5,11.25);
                \draw [short] (12.5,11.25) -- (12.5,14);
                \draw [short] (12.5,14) -- (13.75,16.5);
                \draw [short] (7.5,9) -- (7.5,11.5);
                \draw [short] (7.5,11.5) -- (10,14);
                \draw [short] (10,14) -- (13.75,16.5);
                \draw [short] (15,9) -- (16.25,11.5);
                \draw [short] (17.5,9) -- (16.25,11.5);
                \draw [short] (16.25,11.5) -- (15,14);
                \draw [short] (15,14) -- (13.75,16.5);
                \draw [short] (17.75,14) -- (13.75,16.5);
                \draw [ color={rgb,255:red,255; green,0; blue,0}, ->, >=Stealth] (13,16.75) .. controls (5.25,14.5) and (6,13.75) .. (5.25,7.25) ;
                \draw [ color={rgb,255:red,255; green,0; blue,0}, ->, >=Stealth] (5.75,6.5) -- (7,6.5);
                \draw [ color={rgb,255:red,255; green,0; blue,0}, ->, >=Stealth] (10.25,7.25) -- (8.25,9);
                \draw [ color={rgb,255:red,255; green,0; blue,0}, ->, >=Stealth] (8.25,6.5) -- (9.5,6.5);
                \draw [ color={rgb,255:red,255; green,0; blue,0}, ->, >=Stealth] (8.25,11.75) -- (9.75,13.25);
                \node [font=\LARGE, color={rgb,255:red,255; green,0; blue,0}] at (17,4.75) {\textbf{}};
                \draw [ color={rgb,255:red,255; green,0; blue,0}, ->, >=Stealth] (8,9.5) -- (8,11);
                \draw [ color={rgb,255:red,255; green,0; blue,0}, ->, >=Stealth] (18.25,14.75) -- (14.5,16.75);
                \draw [ color={rgb,255:red,255; green,0; blue,0}, ->, >=Stealth] (10.25,13.25) -- (12,7.25);
                \draw [ color={rgb,255:red,255; green,0; blue,0}, ->, >=Stealth] (13.25,6.75) -- (13.25,8.5);
                \node [font=\LARGE, color={rgb,255:red,255; green,0; blue,0}] at (13.5,7) {\textbf{}};
                \draw [ color={rgb,255:red,255; green,0; blue,0}, ->, >=Stealth] (13.25,9.25) -- (13.25,11);
                \draw [ color={rgb,255:red,255; green,0; blue,0}, ->, >=Stealth] (13.25,11.75) -- (13.25,13.5);
                \draw [ color={rgb,255:red,255; green,0; blue,0}, ->, >=Stealth] (13.25,14) -- (14.75,10);
                \draw [ color={rgb,255:red,255; green,0; blue,0}, ->, >=Stealth] (15.75,9) -- (17,9);
                \draw [ color={rgb,255:red,255; green,0; blue,0}, ->, >=Stealth] (17.75,9.75) -- (17,11.25);
                \draw [ color={rgb,255:red,255; green,0; blue,0}, ->, >=Stealth] (16.5,12.25) -- (15.75,13.75);
                \draw [ color={rgb,255:red,255; green,0; blue,0}, ->, >=Stealth] (15.75,14) -- (17,14);
                \node [font=\large, color={rgb,255:red,255; green,0; blue,0}] at (5,5.5) {[incluir, excluir]};
                \end{circuitikz}
                }%
            
            \label{fig:arbolFiesta1}
        \end{figure}
    \end{center}

    \vspace{-1.2cm}
    Seguimos el camino rojo y para cada nodo vamos calculando sus 2 valores, esto sucede en a lo mas O(n) pues cada nodo puede tener a lo mas n-1 hijos. (las funciones de incluir y excluir tienen que checar a todos sus hijos) pero en el caso de las hojas y en general cuando tienen una cantidad razonable de hijos (hijos "constantes" relativo a n) se hace en O(1).\vspace{.2cm}

    \begin{center}
        \begin{figure}[H]
            \centering
            \resizebox{.55\textwidth}{!}{%  
            \begin{circuitikz}
                \tikzstyle{every node}=[font=\large]
                \draw  (5,6.5) circle (0cm);
                \draw  (5,6.5) circle (0.5cm) node {\LARGE 3} ;
                \draw  (7.5,6.5) circle (0.5cm) node {\LARGE 2} ;
                \draw  (10,6.5) circle (0.5cm) node {\LARGE 1} ;
                \draw  (12.5,6.5) circle (0.5cm) node {\LARGE \textbf{5}} ;
                \draw  (7.5,9) circle (0.5cm) node {\LARGE \textbf{6}} ;
                \draw  (12.5,9) circle (0.5cm) node {\LARGE \textbf{5}} ;
                \node [font=\LARGE] at (8,8.5) {};
                \draw  (7.5,11.5) circle (0.5cm) node {\LARGE \textbf{6}} ;
                \draw  (15,9) circle (0.5cm) node {\LARGE \textbf{12}} ;
                \draw  (17.5,9) circle (0.5cm) node {\LARGE \textbf{0}} ;
                \draw  (16.25,11.5) circle (0.5cm) node {\LARGE \textbf{8}} ;
                \draw  (12.5,11.25) circle (0.5cm) node {\LARGE \textbf{7}} ;
                \draw  (12.5,14) circle (0.5cm) node {\LARGE \textbf{12}} ;
                \draw  (13.75,16.5) circle (0.5cm) node {\LARGE \textbf{30}} ;
                \draw  (10,14) circle (0.5cm) node {\LARGE \textbf{0}} ;
                \draw  (15,14) circle (0.5cm) node {\LARGE \textbf{12}} ;
                \draw  (17.75,14) circle (0.5cm) node {\LARGE \textbf{2}} ;
                \draw [short] (5,6.5) -- (7.5,9);
                \draw [short] (7.5,6.5) -- (7.5,9);
                \draw [short] (10,6.5) -- (7.5,9);
                \draw [short] (12.5,6.5) -- (12.5,9);
                \draw [short] (12.5,9) -- (12.5,11.25);
                \draw [short] (12.5,11.25) -- (12.5,14);
                \draw [short] (12.5,14) -- (13.75,16.5);
                \draw [short] (7.5,9) -- (7.5,11.5);
                \draw [short] (7.5,11.5) -- (10,14);
                \draw [short] (10,14) -- (13.75,16.5);
                \draw [short] (15,9) -- (16.25,11.5);
                \draw [short] (17.5,9) -- (16.25,11.5);
                \draw [short] (16.25,11.5) -- (15,14);
                \draw [short] (15,14) -- (13.75,16.5);
                \draw [short] (17.75,14) -- (13.75,16.5);
                \draw [ color={rgb,255:red,255; green,0; blue,0}, ->, >=Stealth] (13,16.75) .. controls (5.25,14.5) and (6,13.75) .. (5.25,7.25) ;
                \draw [ color={rgb,255:red,255; green,0; blue,0}, ->, >=Stealth] (5.75,6.5) -- (7,6.5);
                \draw [ color={rgb,255:red,255; green,0; blue,0}, ->, >=Stealth] (10.25,7.25) -- (8.25,9);
                \draw [ color={rgb,255:red,255; green,0; blue,0}, ->, >=Stealth] (8.25,6.5) -- (9.5,6.5);
                \draw [ color={rgb,255:red,255; green,0; blue,0}, ->, >=Stealth] (8.25,11.75) -- (9.75,13.25);
                \node [font=\LARGE, color={rgb,255:red,255; green,0; blue,0}] at (17,4.75) {\textbf{}};
                \draw [ color={rgb,255:red,255; green,0; blue,0}, ->, >=Stealth] (8,9.5) -- (8,11);
                \draw [ color={rgb,255:red,255; green,0; blue,0}, ->, >=Stealth] (18,14.75) -- (14.5,16.75);
                \draw [ color={rgb,255:red,255; green,0; blue,0}, ->, >=Stealth] (10.25,13.25) -- (12,7.25);
                \draw [ color={rgb,255:red,255; green,0; blue,0}, ->, >=Stealth] (13.25,6.75) -- (13.25,8.5);
                \node [font=\LARGE, color={rgb,255:red,255; green,0; blue,0}] at (13.5,7) {\textbf{}};
                \draw [ color={rgb,255:red,255; green,0; blue,0}, ->, >=Stealth] (13.25,9.25) -- (13.25,11);
                \draw [ color={rgb,255:red,255; green,0; blue,0}, ->, >=Stealth] (13.25,11.75) -- (13.25,13.5);
                \draw [ color={rgb,255:red,255; green,0; blue,0}, ->, >=Stealth] (13.25,14) -- (14.75,10);
                \draw [ color={rgb,255:red,255; green,0; blue,0}, ->, >=Stealth] (15.75,9) -- (17,9);
                \draw [ color={rgb,255:red,255; green,0; blue,0}, ->, >=Stealth] (17.75,9.75) -- (17,11.25);
                \draw [ color={rgb,255:red,255; green,0; blue,0}, ->, >=Stealth] (16.5,12.25) -- (15.75,13.75);
                \draw [ color={rgb,255:red,255; green,0; blue,0}, ->, >=Stealth] (15.75,14) -- (17,14);
                \node [font=\large, color={rgb,255:red,255; green,0; blue,0}] at (5,17.25) {[incluir, excluir]};
                \node [font=\large, color={rgb,255:red,255; green,0; blue,0}] at (4.75,5.5) {[3, 0]};
                \node [font=\large, color={rgb,255:red,255; green,0; blue,0}] at (7.5,5.5) {[2, 0]};
                \node [font=\large, color={rgb,255:red,255; green,0; blue,0}] at (10,5.5) {[1, 0]};
                \node [font=\large, color={rgb,255:red,255; green,0; blue,0}] at (9.5,9.25) {[6, 6]};
                \node [font=\large, color={rgb,255:red,255; green,0; blue,0}] at (9,11.5) {[12, 6]};
                \node [font=\large, color={rgb,255:red,255; green,0; blue,0}] at (10,15) {[6, 12]};
                \node [font=\large, color={rgb,255:red,255; green,0; blue,0}] at (12.5,5.5) {[5, 0]};
                \node [font=\large, color={rgb,255:red,255; green,0; blue,0}] at (11.75,10) {[5, 5]};
                \node [font=\large, color={rgb,255:red,255; green,0; blue,0}] at (11.5,12) {[12, 5]};
                \node [font=\large, color={rgb,255:red,255; green,0; blue,0}] at (11.5,13.25) {[17, 12]};
                \node [font=\large, color={rgb,255:red,255; green,0; blue,0}] at (15,8) {[12, 0]};
                \node [font=\large, color={rgb,255:red,255; green,0; blue,0}] at (17.75,8) {[0, 0]};
                \node [font=\large, color={rgb,255:red,255; green,0; blue,0}] at (17.5,12) {[8, 12]};
                \node [font=\large, color={rgb,255:red,255; green,0; blue,0}] at (14.5,13) {[24, 12]};
                \node [font=\large, color={rgb,255:red,255; green,0; blue,0}] at (18.25,13.25) {[2, 0]};
                \node [font=\large, color={rgb,255:red,255; green,0; blue,0}] at (13.75,17.5) {[66, 55]};
                \end{circuitikz}
            }%
            \label{fig:arbolFiesta2}
        \end{figure}
    \end{center}

    \vspace{-1.2cm}

    De aqui ya podemos sacar que la respuesta es el maximo entre incluir y excluir de la raiz; pero el problema nos pide tambien la lista asi que vamos a crearla, para esto vamos a hacer un recorrido con bfs para checar por nivel viendo si en cada nodo nos convino incluirlo o excluirlo, ademas, si un nodo es incluido no tenemos que checar a cada uno de sus hijos, en ese caso solo hay que ir a checar a cada uno de sus nietos, finalmente si un valor queda como que da igual (y no tiene restricciones) podemos incluirlo o exlcuirlo, esto puede pasar si hay varios caminos o si el numero es 0.\vspace{.2cm}

    Esto nos puede quedar algo asi: \vspace{.2cm}
    \begin{center}
        \begin{figure}[H]
            \centering
            \resizebox{.55\textwidth}{!}{%  
            \begin{circuitikz}
                \tikzstyle{every node}=[font=\LARGE]
                \draw  (5,6.5) circle (0cm);
                \draw [ color={rgb,255:red,0; green,255; blue,0} ] (5,6.5) circle (0.5cm) node {\LARGE 3} ;
                \draw [ color={rgb,255:red,0; green,255; blue,0} ] (7.5,6.5) circle (0.5cm) node {\LARGE 2} ;
                \draw [ color={rgb,255:red,0; green,255; blue,0} ] (10,6.5) circle (0.5cm) node {\LARGE 1} ;
                \draw [ color={rgb,255:red,0; green,255; blue,0} ] (12.5,6.5) circle (0.5cm) node {\LARGE \textbf{5}} ;
                \draw  (7.5,9) circle (0.5cm) node {\LARGE \textbf{6}} ;
                \draw  (12.5,9) circle (0.5cm) node {\LARGE \textbf{5}} ;
                \node [font=\LARGE] at (8,8.5) {};
                \draw [ color={rgb,255:red,0; green,255; blue,0} ] (7.5,11.5) circle (0.5cm) node {\LARGE \textbf{6}} ;
                \draw [ color={rgb,255:red,0; green,255; blue,0} ] (15,9) circle (0.5cm) node {\LARGE \textbf{12}} ;
                \draw [ color={rgb,255:red,255; green,255; blue,0} ] (17.5,9) circle (0.5cm) node {\LARGE \textbf{0}} ;
                \draw  (16.5,11.25) circle (0.5cm) node {\LARGE \textbf{8}} ;
                \draw [ color={rgb,255:red,0; green,255; blue,0} ] (12.5,11.25) circle (0.5cm) node {\LARGE \textbf{7}} ;
                \draw  (12.5,14) circle (0.5cm) node {\LARGE \textbf{12}} ;
                \draw [ color={rgb,255:red,0; green,255; blue,0} ] (13.75,16.5) circle (0.5cm) node {\LARGE \textbf{30}} ;
                \draw  (10,14) circle (0.5cm) node {\LARGE \textbf{0}} ;
                \draw  (15,14) circle (0.5cm) node {\LARGE \textbf{12}} ;
                \draw  (17.75,14) circle (0.5cm) node {\LARGE \textbf{2}} ;
                \draw [short] (5,6.5) -- (7.5,9);
                \draw [short] (7.5,6.5) -- (7.5,9);
                \draw [short] (10,6.5) -- (7.5,9);
                \draw [short] (12.5,6.5) -- (12.5,9);
                \draw [short] (12.5,9) -- (12.5,11.25);
                \draw [short] (12.5,11.25) -- (12.5,14);
                \draw [short] (12.5,14) -- (13.75,16.5);
                \draw [short] (7.5,9) -- (7.5,11.5);
                \draw [short] (7.5,11.5) -- (10,14);
                \draw [short] (10,14) -- (13.75,16.5);
                \draw [short] (15,9) -- (16.5,11.25);
                \draw [short] (17.75,9) -- (16.5,11.5);
                \draw [short] (16.5,11.5) -- (15,14);
                \draw [short] (15,14) -- (13.75,16.5);
                \draw [short] (17.75,14) -- (13.75,16.5);
                \node [font=\LARGE, color={rgb,255:red,255; green,0; blue,0}] at (17,4.75) {\textbf{}};
                \node [font=\LARGE, color={rgb,255:red,255; green,0; blue,0}] at (13.5,7) {\textbf{}};
                \node [font=\large, color={rgb,255:red,255; green,0; blue,0}] at (5,17.25) {[incluir, excluir]};
                \node [font=\large, color={rgb,255:red,255; green,0; blue,0}] at (4.75,5.5) {[3, 0]};
                \node [font=\large, color={rgb,255:red,255; green,0; blue,0}] at (7.5,5.5) {[2, 0]};
                \node [font=\large, color={rgb,255:red,255; green,0; blue,0}] at (10,5.5) {[1, 0]};
                \node [font=\large, color={rgb,255:red,255; green,0; blue,0}] at (9.5,9.25) {[6, 6]};
                \node [font=\large, color={rgb,255:red,255; green,0; blue,0}] at (9,11.5) {[12, 6]};
                \node [font=\large, color={rgb,255:red,255; green,0; blue,0}] at (10,15) {[6, 12]};
                \node [font=\large, color={rgb,255:red,255; green,0; blue,0}] at (12.5,5.5) {[5, 0]};
                \node [font=\large, color={rgb,255:red,255; green,0; blue,0}] at (11.75,10) {[5, 5]};
                \node [font=\large, color={rgb,255:red,255; green,0; blue,0}] at (11.5,12) {[12, 5]};
                \node [font=\large, color={rgb,255:red,255; green,0; blue,0}] at (11.5,13.25) {[17, 12]};
                \node [font=\large, color={rgb,255:red,255; green,0; blue,0}] at (15,8) {[12, 0]};
                \node [font=\large, color={rgb,255:red,255; green,0; blue,0}] at (17.75,8) {[0, 0]};
                \node [font=\large, color={rgb,255:red,255; green,0; blue,0}] at (17.5,12) {[8, 12]};
                \node [font=\large, color={rgb,255:red,255; green,0; blue,0}] at (14.5,13) {[24, 12]};
                \node [font=\large, color={rgb,255:red,255; green,0; blue,0}] at (18.25,13.25) {[2, 0]};
                \node [font=\large, color={rgb,255:red,255; green,0; blue,0}] at (13.75,17.5) {[66, 55]};
                \draw [ color={rgb,255:red,0; green,0; blue,255}, ->, >=Stealth] (13.25,15.75) -- (8.25,11.75);
                \draw [ color={rgb,255:red,0; green,0; blue,255}, ->, >=Stealth] (8,11) -- (11.75,11);
                \draw [ color={rgb,255:red,0; green,0; blue,255}, ->, >=Stealth] (13.25,11) -- (15.75,11);
                \draw [ color={rgb,255:red,0; green,0; blue,255}, ->, >=Stealth] (16.5,10.5) -- (15.75,9.5);
                \draw [ color={rgb,255:red,0; green,0; blue,255}, ->, >=Stealth] (5.75,6.5) -- (6.75,6.5);
                \node [font=\LARGE, color={rgb,255:red,0; green,0; blue,255}] at (6.75,6) {\textbf{}};
                \draw [ color={rgb,255:red,0; green,0; blue,255}, ->, >=Stealth] (8.25,6.5) -- (9.25,6.5);
                \node [font=\LARGE, color={rgb,255:red,0; green,0; blue,255}] at (16.75,9.5) {\textbf{}};
                \draw [ color={rgb,255:red,0; green,0; blue,255}, ->, >=Stealth] (16,9) -- (16.75,9);
                \draw [ color={rgb,255:red,0; green,0; blue,255}, ->, >=Stealth] (16.75,8.5) -- (5.75,7);
                \draw [ color={rgb,255:red,0; green,0; blue,255}, ->, >=Stealth] (10.75,6.5) -- (11.75,6.5);
                \end{circuitikz}
            }%
            \label{fig:arbolFiesta3}
        \end{figure}
    \end{center}

    \vspace{-1.2cm}

    El camino azul nos dice cuales tenemos que verificar pero hay que recalcar que todos los nodos entran a la fila en algun momento aunque sea para meter a sus hijos, por lo que la complejidad de este paso es O(n) (bfs usualmente es la suma de aristas y vertices pero en arboles tenemos n-1 aristas). \vspace{.2cm}

    Los nodos que agregamos los pinte de verde pero podemos agregarlos a una lista y los que excluyo son o hijos de incluidos o nodos cuyo valor de excluir es mayor que el de incluir y no quedan coloreados (o incluidos). \vspace{.2cm}

    En total entonces el algoritmo tiene complejidad de recorrer el arbol con un recorrido postorden, agregar informacion por cada nodo mas el recorrerlo con bfs, o lo que es lo mismo O(n)*O(1) + O(n) = O(n) y el espacio adicional es O(n) para guardar la lista de nodos que vamos a incluir. (notemos que en el caso de que un nodo tiene muchos hijos estos a su vez ya no pueden tener muchos hijos). \vspace{.2cm}
\end{quote}
\textbf{Sea $G$ una gr\'afica cuyas aristas tienen asignados pesos positivos. Sea $T$ un \'arbol generador de peso m\'inimo de G. Pruebe que existen aristas $e \in T$ y $e' \notin T$ tales que $T - \{e \cup e'\}$ forman un \'arbol de peso mayor o igual que $T$, pero menor o igual a cualquier otro \'arbol generador de $G$, i.e. un segundo \'arbol generador de peso m\'inimo.}\vspace{.2cm}

\textcolor{bibi}{}
\begin{quote}
\end{quote}
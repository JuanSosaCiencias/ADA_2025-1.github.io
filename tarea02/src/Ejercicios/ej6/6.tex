\textbf{Eres un joven cientifico quien acaba de recibir un nuevo trabajo
en un gran equipo de 100 personas (tu eres la persona 101). Un amigo tuyo
en quien confias te dijo que tienes mas colegas honestos que metirosos;
eso es todo lo que te dijo. Un mentiroso es una persona que puede decir
 una verdad o una mentira, mientras que una persona honesta es alguien 
 que siempre dice la verdad. Claro, a ti te gustaría saber quienes son
 los mentirosos y quienes son honestos, asi que decides empezar una 
 investigación. Haces una serie de preguntas a tus colegas, Ya que no
 quieres aparecer sospechoso, decides solo hacer preguntas del tipo 
 "¿Y una persona honesta?" y claro, quieres hacer las menos preguntas 
 posibles. ¿Puedes distinguir a todos tus colegas honestos?. ¿Cúal es la
 mínima cantidad de preguntas que tienes que preguntar en el peor caso? 
 Puedes asumir que tus colegas se conocen muy bien entre ellos, lo
 suficiente para decir si otra persona es un mentiroso o no. (Hint: 
 Agrupa a las personas en pares $(X,Y)$ y pregunta a $X$ si $Y$ es mentiroso
 y después a $Y$ preguntale si $X$ es mentiroso o a los 2 si el otro es
 honest; Analiza las 4 posibles respuestas. Una vez encuentres a una persona
 honesta, puedes usarlo para encontrar a todos los otros. Como desafio, ¿puedes
 resolver el problema con menos de 280 preguntas?.)\\\\
 Generaliza la estrategía de arriba para mostrar que dadas \textit{n} personas
 tal que menos de la mitad son metirosos, puedes separar a los mentirosos de los
 honestos en $\Theta (n)$ preguntas.}\\
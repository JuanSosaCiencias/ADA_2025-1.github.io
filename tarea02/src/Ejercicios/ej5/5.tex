\textbf{Rotar un arreglo significa hacer un corrimiento a la derecha de 
todos los elementos, excepto el último que pasará a la primera posición
 del arreglo. Por ejemplo $A=[1,3,4,5,7,10,14,15,16,19,20,25]$ y después 
 de una rotación $A=[25,1,3,4,5,7,10,14,15,16,19,20]$. Dado un arreglo
 ordenado \textit{A} de n números enteros distintos que ha sido rotado un 
 número desconocido de veces. Diseña un algoritmo de $O(log \, n)$ tiempo
 que encuentre el \textit{k-ésimo} elemento del arreglo. Por ejemplo, $A=
 [15,16,19,20,25,1,3,4,5,7,10,14]$ y $k=5$ la respuesta es 7.}\\
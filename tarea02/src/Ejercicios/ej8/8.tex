\textbf{Permutaciones de Josephus: Supongamos que n personas están 
sentadas alrededor de una mesa circular con $n$ sillas, y que tenemos
un entero positivo $m \leq n$. Comenzando con la persona con etiqueta 1, 
(moviendonos siempre en la dirección de las manecillas del reloj) comenzamos
a remover los ocupantes de las sillas como sigue: Primero eliminiamos la persona
con etiqueta $m$. Recursivamente, eliminamos al \textit{m-ésimo} elemento de los
elementos restantes. Este proceso continua hasta que las $n$ personas han sido
eliminadas. El orden en que las personas han sido eliminadas, se le conoce como
la \textit{(n,m)-permutación de Josephus}. Por ejemplo si $n=7$ y $m=3$, la 
,\textit{(7,3)-permutación de Josephus} es: \{3,6,2,7,5,1,4\}}
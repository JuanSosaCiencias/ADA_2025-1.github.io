\textbf{Se dice que un arreglo A[1,...,n] es k-ordenado si este puede ser dividido en k bloques cada uno de tamaño $\frac{n}{k}$ aproximadamente, tal que todos los elementos en cada bloque son mas grandes que el bloque anterior y mas pequeños que los elementos del bloque siguiente. Los elementos en cada bloque podrían no estar ordenados. Por ejemplo, el siguiente arreglo es 4-ordenado:
\begin{align*}
    1,2,4,3 \,|\, 7,6,5 \,|\, 10,11,9,12 \,|\, 15,13,16,14\\
\end{align*}}


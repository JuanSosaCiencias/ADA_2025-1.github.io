\textbf{Describe un algoritmo que use $\Theta(n^2)$ comparaciones para resolver el problema.}\\

La primer idea es tomar cada contenedor de madera, llenarlo de sake y probar pasarlo hacia cada uno de los de vidrio contenedor $1$ hasta el $n$ de vidrio en el peor de los casos (obviamente tenemos que rellenarlo cada que no sean del mismo tamaño y no lo menciona pero supongo que también vaciar los contenedores de vidrio), esto en el peor de los casos nos toma n unidades de tiempo, ahora nos quedan n-1 contenedores de ambos tipos y repetimos, esto también esta en el orden de n unidades de tiempo y vamos a repetir para cada contenedor de madera, entonces la complejidad es $n+(n-1)+(n-2)\dots+1 = \frac{n(n+1)}{2} \in O(n^2)$ además podemos demostrar que tiene una cota superior si multiplicamos por ejemplo por 2 entonces $c_2=2$ y $\frac{n^2+x}{2} \leq 2*n^2$ y esto demuestra la cota superior $\forall n>1$, ahora tambien si tomamos $c_1=.1$ tenemos que $.1 n^2 \leq \frac{n^2+x}{2}$ $\forall n>1$ entonces demostrando que este algoritmo $\in \Theta(n^2)$.\\
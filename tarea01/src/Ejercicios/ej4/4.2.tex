\textbf{Suponga por obvias razones, que solo nos interesa encontrar los contenedores que les cabe mas sake. Pruebe que esto puede hacerse con a lo mas 2n-2 comparaciones.}\\

Ahora, tomamos nuestro contenedor de madera, y comenzamos a rellenar y vertir en cada uno de los de vidrio, la idea es, buscar uno que le quepa mas que el que elegimos, en el peor de los casos comparamos nuestro botecito de madera con todos los y no fue hasta el ultimo que encontramos uno de vidrio que es mas grande (y no fue igual), afortunadamente ya solo quedan n-1 contenedores de madera así que comenzamos a comparar desde el segundo, en el peor de los casos, comparamos hasta llegar al penúltimo y este también es mas chico que nuestro bote de vidrio, es decir que ya sin comparar podemos saber que el ultimo contenedor de madera sera el que es de igual tamaño al que tenemos de vidrio y este par es el mas grande.\\

Para resumir el algoritmo, tomamos un contenedor de madera, el que sea, comparamos con cada uno de los de vidrio hasta llegar a uno mas grande (esto descarta los que comparaste antes) o acabemos con todos y solo a uno le cupo lo mismo (en cuyo caso acabaste), a la hora de intercambiar eliminamos los de madera que ya hayamos usado o comparado, de manera que seguimos con los que aun pueden ser mas grandes si encontramos uno mas grande otra vez alternamos y ahora usamos ese grande de madera para comparar con los de vidrio; notemos que recorremos el arreglo de los de vidrio hasta que no, tomamos el bote en el índice i+1 y recorremos los de madera 
desde el ultimo que usamos.\\

Como ya mencionamos esto en el peor de los casos recorre todos los de vidrio y después se pone a recorrer desde el segundo de madera hasta el penúltimo de madera (el primero lo usamos para comparar con los n de vidrio y el penúltimo no tiene caso comparar) así usando 2n-2 comparaciones para encontrar la pareja.\\